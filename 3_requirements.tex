%\chapter{System design}

%In this chapter, design of the radio link and component selection are described. During the design process, multiple iteration of component selection, validation and link budget calculation were made, this chapter describes the final result of the design.

\chapter{System requirements}
\section{Communication sessions}
PW-Sat2 was designed to be deployed on \SI{600}{\kilo\meter}, Sun-synchronous, polar orbit. Due to the Earts rotation, ground station coverage is bounded, and communication with the satellite is limited to communication session during satellite overpass.

Simulations performed using Gpredict software \cite{gpredict_website} shown that communication sessions will happen \si{5}-\si{6} times per day, \si{5}-\SI{10}{\minute} each. This summarizes to the total communication time to average \SI{30}{\minute} per day. Typical radio coverage of the satellite is shown in the figure \ref{gpredict_pass}.

\begin{figure}[H]
    \centering
    \includegraphics[width=0.8\paperwidth]{img/3/gpredict_pass.png}
    \caption{PW-Sat2 pass in Gpredict software.}
    \label{gpredict_pass}
\end{figure}


\section{Data link budget}
PW-Sat2 have couple of data sources to be sent to the ground:
\begin{itemize}
    \item housekeeping telemetry (satellite health information, such as temperatures, bus and solar panel voltages and currents, subsystem status etc. Telemetry is \SI{230}{\byte} frame and should be transmitted every \SI{1}{\minute} (average \SI{30}{\bps}),
    \item archival housekeeping telemetry (gathered during orbit, when no ground station is in range), telemetry should be downloaded in \SI{5}{\minute} step from the whole operational time, which sums up to \SI{64}{\kilo\byte} per day (average \SI{300}{\bps}),
    \item experiment data, which are started by the operator from the ground and each of them is planned to run once a week (Sun Sensor -  \SI{32}{\kilo\byte}, RadFET - \SI{4}{\kilo\byte}, Cameras (10 photos) - \SI{300}{\kilo\byte}) - it sums up to \SI{336}{\kilo\byte} per week, average of \SI{220}{\bps},
    \item Sail deployment procedure, which transmits experiment data on-line (during the experiment). Estimated data to be transmitted (sail deployment indicator, photos and gyroscopes) is about \SI{1}{\kbps}.
\end{itemize}

Data throughput requirement for PW-Sat2 mission sums up to about \SI{1}{\kbps}.

\section{Frequency band selection}
The most common radio bands used in CubeSat designs are VHF, UHF and S-band. S-band is usually used when high data rate (~\SI{10}{\Mbps}) are necessary. Typical designs for low-rate data link are full-duplex combo or simplex VHF/UHF radio.

PW-Sat1 \cite{pwsat1_website_ska} used full-duplex VHF-downlink and UHF-uplink radio. During its mission, operation team was reporting uplink stability issues, which was caused by very low Signal to Noise ratio on the satellite. The design team narrowed down the issue to very high level of interference in the UHF band on the orbit, which probably is caused by the high power signal sources on the ground, such as radars.

The selected bands for PW-Sat2 operation were either simplex VHF or full-duplex: VHF-up, UHF-down.


\section{Communication link parameters}
Selected satellite radio module (as described in \autoref{section:comm_design}) imposes the modulation and data packets used in the communication. This section briefly describes used modulations, frame formats and implications to the ground segment of the system.

\subsection{Downlink modulation}
Downlink signal is modulated using Binary Phase Shift Keying (BPSK). The data rate can be changed dynamically by the spacecraft On-Board Computer, in range \si{1.2} - \SI{9.6}{\kbps}, allowing to improve the link quality when necessary. Baseband signal is also filtered using Raised Root Cosine filter, to reduce side lobes power. Signal bandwidth varies between \SI{2.4}{\kHz} (for \SI{1.2}{\kbps}) and \SI{19.2}{\kHz} (for \SI{9.6}{\kbps}).

% TODO: rysunek BPSK

\subsection{Uplink}
Uplink modulation is two-staged: first, data is modulated using Audio Frequency Shift Keying (chaning 0s and 1s to wave of frequencies, in order, \SI{1200}{\hertz} and \SI{2200}{\hertz}), later to be Frequency Modulated to the RF carrier (with frequency deviation of $\pm\SI{5}{\kilo\hertz}$).

% TODO: schemat modulatora

\subsection{Frame format}
Physical layer data format is reduced-functionality AX.25 packet \cite{AX25_standard}. Only connectionless transmission mode and UI frames are supported. Packet framing is the same as in SwissCube CubeSat \cite{SwissCube_AX25}.

Basic frame format is shown in the table \ref{AX25_frame}. All the fields except flags are bit-stuffed to ensure that the \textit{Flag} field does not appear in the data: if there are six '1' bits to be send, transmitter inserts '0' bit before the last one. Adressing in this system in inherent, but unused - this is point-to-point connection, therefore adresses are fixed. Frame-Check Sequence is a CRC (CITT standard) of the whole frame (without \textit{Flags}). \textit{Information Field} is variable-length, between \si{4} and \SI{256}{\byte} length is the place for the actual data transmitted by the On-Board Computer.

\begin{table}
\small
\centering
\caption{AX.25 frame format}
\label{AX25_frame}
\arrayrulecolor{black}
\begin{tabular}{l|c|c|c|c|c|c|c|c|} 
\hhline{~|-|----|-|-|-|}
\multirow{2}{*}{}                                                              & \multirow{2}{*}{Flag } & \multicolumn{4}{c|}{AX.25 Transfer Frame Header (128 bits)}                                                                                                                                                                                   & {\cellcolor[rgb]{0.753,0.753,0.753}}                                                                                                                  & \multirow{2}{*}{\begin{tabular}[c]{@{}c@{}}Frame-\\Check\\Sequence\end{tabular}} & \multirow{2}{*}{Flag}  \\ 
\hhline{~|~|-|-|-|-|>{\arrayrulecolor[rgb]{0.753,0.753,0.753}}->{\arrayrulecolor{black}}|~|~|}
                                                                               &                        & \begin{tabular}[c]{@{}c@{}}Destination\\Address\end{tabular} & \begin{tabular}[c]{@{}c@{}}Source\\Address\end{tabular} & \begin{tabular}[c]{@{}c@{}}Control\\Bits\end{tabular} & \begin{tabular}[c]{@{}c@{}}Protocol\\Identifier\end{tabular} & \multirow{-2}{*}{{\cellcolor[rgb]{0.753,0.753,0.753}}\begin{tabular}[c]{@{}>{\cellcolor[rgb]{0.753,0.753,0.753}}c@{}}Information\\Field\end{tabular}} &                                                                                  &                        \\ 
\hline
\multicolumn{1}{|c|}{\begin{tabular}[c]{@{}c@{}}Length\\{[}bits]\end{tabular}} & 8                      & 56                                                           & 56                                                      & 8                                                     & 8                                                            & {\cellcolor[rgb]{0.753,0.753,0.753}}32-2048                                                                                                           & 16                                                                               & 8                      \\ 
\hline
\multicolumn{1}{|c|}{Value}                                                    & 01111110               & PWSAT2-0                                                     & PWSAT2-0                                                & 00000011                                              & 11110000                                                     & {\cellcolor[rgb]{0.753,0.753,0.753}}                                                                                                                  & CRC-CITT                                                                         & 01111110               \\
\hline
\end{tabular}
\end{table}


Additionally, downlink data is scrambled using G3RUH scrambling polynomial to maximize randomness and ensure proper bit synchronization.
