\chapter{System design}

In this chapter, component selection and design is described. 

\section{Space segment}
Space segment is a part of the satellite communication system that resides on the satellite itself. It is divided into two main parts: communication subsystem (COMM) and antennas (ANTs) connected with two coaxial cables.

Space segment is critical in system operation and reliability - there is no possibility to carry out any repairs. It is exposed to the space environment - thermal cycling, cosmic radiation and vacuum.

Because of the mentioned requirements it was decided to choose commercially available and flight-proven Cubesat components to increase the reliability of the system.

\subsection{Frequency band selection}
Data throughput requirements and communication session scheme were described in the \ref{introduction} chapter, but it summarises to about \SI{1}{\kilo\byte / \second}.

As mentioned in section \ref{???} the most common radio bands used in Cubesat designs are VHF, UHF and S-band. S-band is usually used when high data rate (~\SI{10}{\mega\bits / \second}) are necessary. Typical designs for low-rate data link are full-duplex combo or simplex VHF/UHF radio.

PW-Sat1 (first polish satellite, built on Warsaw University of Technology) - (link)??? used full-duplex VHF-downlink and UHF-uplink radio. During its mission, uplink stability was an issue. The cause was narrowed down to very high level of noise and interference in the UHF band on the orbit, which probably is caused by over-the-horizon radars. Therefore, UHF-uplink was discarded.

Concluding, the selected band for operation were either simplex VHF or VHF-up, UHF-down full-duplex.

\subsection{Communication subsystem}
Communication subsystem of the satellite is responsible of transmitting and receiving radio signals, modulating/demodulating them and providing data link for the On Board Computer. It has to be compatible with selected mechanical configuration, available data interfaces and the antennas to be installed.

Cubesat standard, with which PW-Sat2 is compliant, defines PC/104 connector, which is the main data bus for all satellite subsystems. Radio should use \iic bus to send and receive data from the on-board computer.

When the subsystem was ordered, the choice of available products was very limited, and the only radio which was compliant with above mentioned requirements was \texttt{ISIS VHF uplink/UHF downlink Full Duplex Transceiver} shown in the figure \ref{ISIS_TRXvU}.

    \begin{figure}[H]
        \centering
        \includegraphics[width=0.4\paperwidth]{img/2/ISIS-radio-UHF-VHF-min.png}
        \caption{ISIS VHF uplink/UHF downlink Full Duplex Transceiver. Source: \cite{???}}
        %%% https://www.isispace.nl/product/isis-uhf-downlink-vhf-uplink-full-duplex-transceiver/
        \label{ISIS_TRXvU}
    \end{figure}
        
Basic characteristics:

\begin{tabular}{c|c}
     \textbf{downlink} & \textbf{uplink} \\ \hline
     \multicolumn{2}{c}{dual-\iic communication standard} \\
     \multicolumn{2}{c}{AX.25 frame format} \\
     \si{430}-\SI{450}{\MHz} frequency range & \si{140}-\SI{150}{\MHz} frequency range \\
     \SI{0.5}{\watt} downlink power & \SI{-98}{\dBm} sensitivity for \si{10^-5}~BER \\
     \si{1.2} - \SI{9.6}{\kilo\bit / \second} bitrate & \SI{1.2}{\kilo\bit / \second} bitrate \\ 
     BPSK modulation with G3RUH scrambling & AFSK \\ 
\end{tabular}


